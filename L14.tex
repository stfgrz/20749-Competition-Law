\setcounter{chapter}{3}
\setchapterabstract{
This chapter explores the concept of control as a prerequisite for mergers and acquisitions under EU competition law, focusing on the conditions and implications of acquiring "decisive influence" over another undertaking. The EU Merger Regulation (Reg. 139/2004) defines concentrations as transactions that result in a lasting change in market structure through mergers, acquisitions, or the creation of joint ventures with operational autonomy. Control can be sole or joint, and may be exercised de jure (via legal rights like voting shares) or de facto (through practical influence, such as voting patterns or dispersed shareholdings). The regulation excludes internal restructurings and non-full-function joint ventures but captures changes in the quality of control, such as shifts between sole and joint control. Full-function joint ventures must meet operational, financial, and market criteria to be considered concentrations. While minority shareholdings and non-controlling transactions fall outside the regulation's direct scope, they may still be scrutinized under Articles 101 or 102 TFEU for potential anticompetitive effects. This nuanced approach underscores the EU’s balance between market oversight and operational flexibility in regulating concentrations.}
\chapter{Acquisition of Control}
\vspace{-1.5cm}

\setcounter{chapter}{14}
{\chaptoc\noindent\begin{minipage}[inner sep=0,outer sep=0]{0.9\linewidth}\section*{Recap}\end{minipage}}

\Definition{
Mergers or «concentrations» can be thought as the result of a process according to which one undertaking may strengthen its position on the market (not by internal growth but) by drawing (acquiring, embedding absorbing) from third parties assets (external growth).
}{Concentrations}

To have a concentration there must be a transfer, i.e. a change of control (ownership, possession, right to use) on a lasting basis of the assets/business (asset or business with a market presence, to which a market turnover can be clearly attributed) of one or more undertakings.

\section{Article 3, Reg. 139/2004}
    
    \begin{quote}
        (2) For the achievement of the aims of the Treaty, Article 3(1)(g) gives the Community the objective of instituting a system ensuring that competition in the internal market is not distorted. [...]
    \end{quote}

    \begin{quote}
        (3) The completion of the internal market and of economic and monetary union, the enlargement of the European Union and the lowering of international barriers to trade and investment will continue to result in major corporate reorganisations, particularly in the form of concentrations.
    \end{quote}

\newpage

    \subsection{Legal Definition}

        \textbf{A concentration shall be deemed to arise where a change of control results from:}
        \begin{enumerate}
            \item The merger of two or more previously independent undertakings or parts of undertakings; or
            \item The acquisition, by one or more persons already controlling at least one undertaking, or by one or more undertakings, whether by purchase of securities or assets, by contract or by any other means, of direct or indirect control of the whole or parts of one or more other undertakings.
        
                \textbf{Control shall be constituted by rights, contracts, or any other means which, either separately or in combination, confer the possibility of exercising decisive influence on an undertaking,} in particular by:
                \begin{enumerate}[label=\roman*.]
                    \item Ownership or the right to use all or part of the assets of an undertaking;
                    \item Rights or contracts which confer decisive influence on the composition, voting, or decisions of the organs of an undertaking.
                \end{enumerate}
            \item The creation of a joint venture performing on a lasting basis.
        \end{enumerate}

    \subsection{Nature of the Article}

        \begin{itemize}
            \item According to Article 3 of the Merger Regulation, a concentration only covers operations where a change of control in the undertakings concerned occurs on a \textbf{lasting basis}. The preamble to the Merger Regulation further explains that the concept of concentration is intended to relate to operations which bring about a \textbf{lasting change in the structure of the market}.
            \item Because the test is centred on the concept of \textbf{control}, the existence of a concentration is determined by \textbf{qualitative} rather than \textbf{quantitative criteria}.
            \item While mergers do not raise specific issues in terms of control (the acquiring or new entity will acquire control over the business of the merging firm or firms), the \textbf{second paragraph of Article 3} must be studied carefully.
        \end{itemize}

    \subsection{Notion of the Article}

        \begin{enumerate}
            \item \textbf{Subject} \(\Rightarrow\) Person/s Undertaking

                The acquirer should be an undertaking. However, control may also be acquired by a person, when that person already controls at least another undertaking or, alternatively, by a combination of persons and undertakings. The term ‘person’ extends to public bodies and private entities, as well as natural persons. \sn{\Note{Acquisitions of control by natural persons are only considered to bring about a lasting change in the structure of the undertakings concerned if those natural persons carry out further economic activities on their own account or if they control at least another undertaking.}}

            \item \textbf{Means} \(\Rightarrow\) How control is acquired

                Control is defined as the possibility of exercising decisive influence on an undertaking. It’s not necessary to show that the decisive influence will be actually exercised. However, the possibility of exercising that influence must be effective. Whether an operation gives rise to an acquisition of control depends on a number of legal and/or factual elements. The most common means for the acquisition of control is the acquisition of shares, possibly combined with a shareholders' agreement in cases of joint control, or the acquisition of assets.\sn{\Note{Acquisition, but also long term agreements for the lease of the business, giving the acquirer control over the management and the resources, despite the fact that property rights or shares are not transferred.}}

            \item \textbf{Object} \(\Rightarrow\) Target

                The target is one or more (or also parts of), undertakings which constitute legal entities, or the assets of such entities, or only some of these asset. Why more? (to avoid parceling the transaction in several smaller transaction each under the notification thresholds, i.e.: gaming, in order not to notify). Why some of these asset ? A part of an undertaking, i.e. a business with a market presence, to which a market turnover can be clearly attributed. A transaction confined to intangible assets such as brands, patents or copyrights may also be considered to be a concentration if those assets constitute a business with a market turnover.
        \end{enumerate}

        \subsubsection{Clarifications}

            \Remark{No change of control inside the same economic entity}

            \begin{itemize}
                \item \textbf{Internal Restructuring}: a concentration within the meaning of the Merger Reg. is limited to changes in control. An internal restructuring within a group of companies does not constitute a concentration (e.g. a merger between two subsidiaries of the same parent company is not a concentration).
                \item \textbf{State-Owned Enterprises}: An exceptional situation exists where both the acquiring and acquired undertakings are companies owned by the same State (or by the same public body or municipality). In this case, whether the operation is to be regarded as an internal restructuring depends in turn on the question whether both undertakings were formerly part of the same economic unit. 
            \end{itemize}

            Where the undertakings were part of different economic units having an independent power of decision, the operation will be deemed to constitute a concentration and not an internal restructuring

\section{Sole Control}

    Sole control is acquired if one undertaking alone can exercise decisive influence on an (other) undertaking. No problem when you are acquiring ownership of an undertaking (or a business/assets from another undertaking). But what if you are acquiring shares?

    \noindent
    Two general situations in which an undertaking has sole control can be distinguished:

        \begin{enumerate}
            \item The solely controlling undertaking enjoys the power to determine the strategic commercial decisions of the other undertaking. This power is typically but not exclusively achieved by the acquisition of a majority of voting rights.
            \item Only 1 shareholder is able to veto strategic decisions of an undertaking, but this shareholder does not have the power, on his own, to impose such decisions\sn{defined \textit{negative sole control}}. In these circumstances, a single shareholder possesses the power to block the adoption of strategic decisions\sn{\Note{No other shareholders enjoy the same level of influence.}}. As a consequence, the shareholder enjoying \textit{negative sole control} does not have to cooperate with other shareholders in determining the strategic behaviour of the controlled undertaking.

                \Example{A has 50\% of a company, while the rest is spread between 5 other shareholders (owning 10\% each one).}
                
        \end{enumerate}

    \subsection{\textit{De Iure} Control}

        Sole control is normally acquired on a legal basis where an undertaking acquires a majority of the voting rights that can be exercised in a general meeting of a corporation (pay attention! Majority of voting rights can be $\neq$ from majority of the stock capital, if for example the corporation issued “non-voting” shares and/or multiple voting shares and/or increased voting rights).
        
        Where the company statutes require a supermajority for strategic decisions, the acquisition of a simple majority of the voting rights may not confer the power to determine strategic decisions

            \Example{s1 has 60\% of the voting rights, but the bylaws of the corporation require a 80\% supermajority for certain strategic decisions}
            
        Even in the case of a minority shareholding, \textit{de jure} sole control may occur in situations where specific rights are attached to this shareholding.

        \Remark{
        These may be preferential shares to which special rights are attached enabling the minority shareholder to determine the strategic commercial behaviour of the target company, such as the power to appoint more than half of the members of the board.
        }

    \subsection{\textit{De Facto} Control}

        A minority shareholder may also be deemed to have sole control on a de facto basis. This is the case where the minority shareholder is highly likely to achieve a majority at the shareholders' meetings, given the level of its shareholding and the evidence resulting from the presence of shareholders in the shareholders' meetings in previous years. 
        
        Based on the past voting pattern, the Commission will carry out a prospective analysis and consider foreseeable changes of the shareholders' presence which might arise in future. 
        
        \Remark{
        The Commission will further analyse the position of other shareholders and assess their role: criteria are whether the remaining shares are widely dispersed, whether other important shareholders have structural, economic or family links with the minority shareholder or whether other shareholders have a strategic or a purely financial interest in the target company. 
        }
        
        Where, based on its shareholding, the historic voting pattern at the shareholders' meeting and the position of other shareholders, a minority shareholder is likely to have a stable majority of the votes at the shareholders' meeting, then that large minority shareholder is taken to have sole control (this happened, for example with Mediobanca and Generali; it was deemed to happen also in Vivendi Telecom

\newpage
    \subsection*{Mock Exam \#1}
    
        \Example{
        “Viva”, a French telecom corp., buys 24\% of the stock capital of the Italian listed corporation “Telec”. However, given that Telec issued some saving shares (shares that do not give the right to vote in the shareholding’s meeting), Viva holds 30\% of the voting rights of Telec. \\
        Average attendance at the general shareholders’ meeting of Telec in the last years has been 50-60\% of the voting share capital. \\
        Two attending shareholders are also merchant banks assisting Viva in its financial/industrial investments. \\
        Many attending shareholders are small pension funds and small Italian institutional investors holding each less than 0,2\% of the share capital each.
        }

\section{Joint Control}

    Joint Control exists where 2 or more undertakings have together the possibility of exercising decisive influence over another undertaking, I.E.: each party has the power to block actions which determine the strategic commercial behaviour of that undertaking.

    Unlike sole control, which confers upon a specific shareholder the power to determine the strategic decisions in an undertaking, JC is characterized by the possibility of a deadlock situation resulting from the power of each of 2 or more parent companies to reject proposed strategic decisions. 
    
    It follows, therefore, that in order to run the company:
    \begin{enumerate}
        \item these parent companies must reach a common understanding in determining the policy of the joint venture and
        \item they are required to cooperate on a longstanding base.
    \end{enumerate}

    The clearest form of JC exists where there are only 2 parent companies which share equally the voting rights in the controlled company (50/50 situation). In this case, it is not necessary for a formal agreement to exist between them. Equality may also be achieved where both parent companies have the right to appoint an equal number of members to the decision-making bodies of the undertaking. \\

    JC may ALSO exist where there is no equality between the 2 shareholders in votes or in representation in decision-making bodies. This is the case where minority shareholders have additional rights which allow them to veto decisions which are essential for the strategic commercial behaviour of the company. 
    
    These veto rights may be set out in the statute (bylaws) of the company or conferred by agreement between its parent companies. The veto rights themselves may operate by means of a specific quorum required for decisions taken at the shareholders' meeting or by the board of directors to the extent that the parent companies are represented on this board.

\newpage
    \subsection*{Mock Exam \#2}

        \Example{
        A owns 30\% and B 70\% of the share capital of Zegna. Zegna’s bylaws require that all the strategic decisions taken by the governing bodies of the firm must have a supermajority of 75\% of the share capital. \\
        \textbf{Analysis}: No shareholder enjoy a decisive influence (sole control), and A and B can individually block each decision of Zegna, i.e. have a veto right, and thus they jointly control Zegna.
        }

        The same applies if the bylaws of a company are silent but a shareholders’ agreement assigns 4 members of the board of directors to A, 3 to B out of a total of 11, and requires that strategic decisions must be taken with a 9/11 super majority.

    \subsection{Reach of veto rights}

        These veto rights must be related to strategic decisions on the business policy of the controlled company.
        
        Veto rights conferring joint control typically include decisions on issues such as:
        \begin{enumerate}[label=\alph*.]
            \item the budget,
            \item the business plan,
            \item major investments,
            \item the appointment of senior management (and executive directors).
        \end{enumerate}
        
        To acquire joint control, it is not necessary to have all the veto rights mentioned above. It may be sufficient that only some, or even one such right, exists. Whether or not this is the case depends upon the precise content of the veto right itself and the importance of this right in the context of the specific business of the controlled company.
        
        The crucial element is that the veto rights are sufficient to enable the parent companies to exercise such influence in relation to the strategic business behaviour of the undertaking.
        
        The acquisition of joint control, thus, does not require that the acquirer has the power to exercise decisive influence on the day-to-day running of an undertaking.

    \subsection{Changes in the quality control}

        The Merger Regulation covers not only operations resulting in the acquisition of sole or joint control, but also operations leading to changes in the quality of control.

        \begin{itemize}
            \item First, such a change in the quality of control, resulting in a concentration, occurs if there is a change between sole and joint control (and vice versa).
            \item Second, a change in the quality of control occurs between joint control scenarios before and after the transaction if there is an increase in the number or a change in the identity of controlling shareholders.
            \Example{If A and B were already two joint controllers, and C also acquires joint control, C has to notify. If D acquires the participation of B (i.e., there is a “replacement”), D has to notify.}
            
            \item However, there is no change in the quality of control if a change from negative to positive sole control occurs. Such a change affects neither the incentives of the negatively controlling shareholder nor the nature of the control structure, as the controlling shareholder did not necessarily have to cooperate with specific shareholders at the time when it enjoyed negative control.
        \end{itemize}
        
        In any case, changes in the level of shareholdings of the same controlling shareholders, without changes in the powers they hold in a company and the composition of the control structure of the company, do not constitute a change in the quality of control and therefore are not a notifiable concentration.

\section{Joint Ventures}

    \Definition{
    A Joint Venture (JV) is:
    \begin{enumerate}[label=(\alph*)]
        \item An undertaking;
        \item Created by two or more undertakings (the parent companies); where
        \item The parent companies are reciprocally independent; and
        \item Jointly control the venture.
    \end{enumerate}
    A joint venture is a concentration when it is \textit{full function}.
    }{Joint Venture}

    The creation of a Joint Venture (JV) performing on a lasting basis all the functions of an autonomous economic entity (so called full-function joint venture) shall constitute a concentration within the meaning of the Merger Reg.

    The fact that a joint venture may be a full-function undertaking and therefore economically autonomous from an operational viewpoint does not mean that it enjoys autonomy as regards the adoption of its strategic decisions. Otherwise, a jointly controlled undertaking could never be considered a full-function joint venture and therefore the condition laid down in the Merger Reg. would never be complied with.

    \Remark{It is sufficient for the criterion of full-functionality if the joint venture is autonomous in operational respect. In other words: if the PC identify the main goals, but the joint venture is free to decide how to reach them, we can say that the joint venture is autonomous.}

    The following sections define what we mean by \textit{full function}.

\newpage
    \subsection{Conditions to be a Joint Venture}

        \subsubsection{Ioint ventures must have sufficient resources to operate independently on a market}

            Full function character essentially means that a joint venture must operate on a market, performing all the functions normally carried out by undertakings operating on the same market. 
            
            In order to do so the joint venture must have a management dedicated to its day-to-day operations and access to sufficient resources including finance, staff, and assets (tangible and intangible) in order to conduct on a lasting basis its business activities within the area provided for in the joint venture agreement.

        \subsubsection{Activities carried out by the joint venture must go beyond one specific function for the parents}

            A joint venture is not full-function if it only takes over one specific function within the parent companies' business activities without its own access to or presence on the market. This is the case, for joint ventures limited to R\&D or to production for their parents. 
            
            Such joint ventures are auxiliary to their parent companies' business activities. This is also the case where a joint venture is essentially limited to the distribution or sales of its parent companies' products and, therefore, acts principally as a sales agency.

        \subsubsection{Sale/purchase relations with the parents}

            The presence of the parent companies in upstream or downstream markets is a factor to be considered in assessing the full-function character of a Joint Venture (JV) where this presence results in substantial sales or purchases between the parent companies and the joint venture.

            The fact that, for an initial start-up period (\(< 3\) years), the joint venture relies almost entirely on sales to or purchases from its parent companies does not affect its full-function character: such a start-up period may be necessary to establish the joint venture on a market.
            
            Where sales from the joint venture to the parent companies will be maintained on a lasting basis, the essential question is whether, regardless of these sales, the joint venture could play an active role on the market and be considered economically autonomous from an operational viewpoint. 
            
            In this respect, the relative proportion of sales made to its parents compared with the total production of the joint venture is an important factor. 

            If the joint venture achieves \(> 50\%\) of its turnover with third parties, this will typically be an indication of full-functionality.

        \subsubsection{Operation on a lasting basis}

            Finally, the joint venture must be intended to operate on a lasting basis. The fact that the parent companies commit to the joint venture the resources described above normally demonstrates that this is the case. 
            
            By contrast, the joint venture will not be considered to operate on a lasting basis where it is established for a short finite duration or a single “one shot” business. This would be the case where a joint venture is established in order to construct a specific project such as a power plant, but it will not be involved in the operation of the plant once its construction has been completed.

\newpage
\section{Transactions not Leading to the Acquisition of Control}

    \begin{itemize}
        \item The acquisition of a minority shareholding that does not lead to (joint or sole) control;
        \item The power to appoint one or more members of the board (without having any veto right leading to joint control);
        \item The formation of a joint venture which is not full function.
    \end{itemize}

    are not caught by the Merger Reg. 
    
    Still, all those transactions may lead to a restriction of competition if they facilitate collusion, or at least a quiet life equilibrium via the commonality of financial interests (or decisional background). This is possible when the parties are competitors on the same product/geographic market. For example, Coca Cola acquiring a 20\% stake in its competitor Virgin Cola and appointing one member of Virgin Cola’s board of directors.

    In some circumstances, it is possible to apply Article 101 or 102 to these ”structural” transactions, i.e. when all the pre-requisites of those two provisions are satisfied. The problem is that this kind of enforcement is only an ex post one (it does not prevent those situation, which are then very difficult to remedy), and it does not cover all the possible dangerous situations (for example, you may become shareholder of your rival without any direct or indirect agreement). 
    
    This is the reason why some Member State introduced special provisions dealing with some of these problems (in Italy, starting from 2012, the legislator introduced a complete ban to direct interlocking directorates in the banking/financial/insurance sectors). 
    
    In EU, a proposal to enlarge the scope of the merger regulation to minority shareholding has been widely discussed, but eventually abandoned because the costs implied by the change in the law have been considered greater than the attached benefits.


