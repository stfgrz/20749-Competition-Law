\setcounter{chapter}{4}
\setchapterabstract{This chapter outlines the competitive assessment of mergers under EU Competition Law, focusing on horizontal and non-horizontal mergers. Horizontal mergers, involving competitors, are evaluated for risks like market power increases or coordination facilitation. Factors include market shares, concentration thresholds (e.g., Herfindahl-Hirschman Index), countervailing buyer power, entry barriers, and efficiency defenses.  Non-horizontal mergers, including vertical (supply chain) and conglomerate (complementary products) mergers, are generally less risky but may lead to foreclosure or coordination. The Illumina/GRAIL case illustrates innovation concerns in vertical mergers.  The EU Commission employs a "counterfactual" analysis to compare market conditions with and without the merger. Mergers that ensure competition or create efficiencies are allowed, while those significantly harming competition are rejected.}
\chapter{Vertical Foreclosure}
\vspace{-1.5cm}

\setcounter{chapter}{15}
{\chaptoc\noindent\begin{minipage}[inner sep=0,outer sep=0]{0.9\linewidth}\section*{Rationale}\end{minipage}}

    The antitrust “mantra” is that effective competition brings benefits to consumers, such as low prices, high quality products, a wide selection of goods and services, and innovation. Thus, through an ex ante appraisal of mergers, the goal is to prevent mergers that would be likely to deprive customers of these benefits by significantly increasing the market power of firms. 
    
    By ‘increased market power’ we mean the ability of one or more firms to profitably increase prices, reduce output, choice or quality of goods and services, decrease the innovation rate, or otherwise negatively influence parameters of competition. Both suppliers and buyers can have market power. 
    \noindent
    However, here we will focus on suppliers’ market power.

\section*{Goal - Article 2, Reg. 139/2004}

    Concentrations shall be appraised with a view to establishing whether they are “compatible with the common market,” taking into account:

    \begin{enumerate}[label=(\alph*)]
        \item The need to maintain and develop competition within the common market in view of the structure of all the markets concerned and the actual or potential competition from undertakings located either within or outside the Community;
        \item The market position of the undertakings concerned and their economic and financial power, the alternatives available to suppliers and users, their access to supplies or markets, any legal or other barriers to entry, supply and demand trends for the relevant goods and services, the interests of the intermediate and ultimate consumers, and the development of technical and economic progress, provided that it is to consumers' advantage and does not form an obstacle to competition.
    \end{enumerate}
    
    \Remark{
    A concentration which would not significantly impede effective competition in the common market or in a substantial part of it, in particular as a result of the creation or strengthening of a dominant position, shall be declared compatible with the common market.
    }
    
    \textit{(Vice versa, a concentration which would significantly impede effective competition shall be declared incompatible with the common market).}

\section{Merger Appraisal Methodology}

    \subsection{The counterfactual}

        In assessing the competitive effects of a merger, the Commission compares the competitive conditions that would result from the notified merger with the conditions that would have prevailed without the merger. 
        
        In most cases the competitive conditions existing at the time of the merger constitute the relevant comparison for evaluating the effects of a merger. However, in some circumstances, the Commission may take into account future changes to the market that can reasonably be predicted. 
        
        \Remark{It may, in particular, take account of the likely entry or exit of firms if the merger did not take place when considering what constitutes the relevant comparison. As we will see later on, this last criterion is very important in case of concentrations involving a failing firm.}

        And the time horizon? Short term, i.e. 1-2 years ? Well, many say you should also consider the effects on the structure of the market in the longer run.

\section{Horizontal Mergers}

    The Commission must take into account any significant impediment to effective competition likely to be caused specifically by a concentration. The creation or the strengthening of a dominant position is a primary form of such competitive harm. \\

    To help firms understand the reasoning behind merger enforcement, the Commission issued the \href{https://drive.google.com/drive/folders/1AlOhjGwM_ZBajGQE6hfb-pc5aUMtSGW9?usp=share_link}{Horizontal Merger Guidelines}. The general idea is that mergers are forbidden when they create excessive market power, i.e., when because of the merger the 3 main constraints:

        \begin{enumerate}
            \item actual competition
            \item potential competition
            \item countervailing buyer power
        \end{enumerate}

    are not sufficient to ensure effective competition (i.e. the well functioning of the markets). However, as we will see, some counter-arguments can be offered to the Commission, in particular in terms of efficiency enhancing effects and/or of failing firm defence.

    \subsection*{Counter-Arguments}

        \subsubsection{Actual Competition - \textcolor{BrickRed}{1° Constraint}}

            The approach to market shares and concentration thresholds.

        \subsubsection{Theory of Harm}

            The likelihood that a merger would have anticompetitive effects in the relevant markets. A distinction between unilateral v. coordinated effects must be drawn: In fact, horizontal mergers may significantly impede effective competition:
            \begin{enumerate}[label=\roman*.]
                \item by eliminating important competitive constraints on one or more firms, which consequently would have increased market power, without resorting to coordinated behaviour (\textbf{non-coordinated effects});
                \item by changing the nature of competition in such a way that firms that previously were not coordinating their behaviour, are now significantly more likely to coordinate and raise prices or otherwise harm effective competition. A merger may also make coordination easier, more stable or more effective for firms which were coordinating prior to the merger (\textbf{coordinated effects}).
            \end{enumerate}

        \subsubsection{Countervailing Buyer Power/Demand - \textcolor{BrickRed}{2° Constraint}}

            The likelihood that buyer power would act as a countervailing factor.

        \subsubsection{Potential Competition - \textcolor{BrickRed}{3° Constraint}}

            The likelihood that entry would maintain effective competition in the relevant markets.

        \subsubsection{Efficiency Defence - \textcolor{ForestGreen}{1° Defence}}

            The likelihood that efficiencies would act as a factor counteracting the harmful effects on competition which might otherwise result from the merger.

        \subsubsection{Failing Firm Defence - \textcolor{ForestGreen}{2° Defence}}

            The conditions for a failing firm defence.

    \subsection{Market Concentration}\label{AC}

        \subsubsection{Market Shares}

            Market shares and concentration levels provide useful first indications of the market structure and of the competitive importance of both the merging parties and their competitors. Normally, the Commission uses current MS in its competitive analysis. However, current MS may be adjusted to reflect reasonably certain future changes, for instance in the light of exit, entry or expansion. Post-merger MS are calculated on the (strong) assumption that the post-merger combined MS of the merging parties is the sum of their pre-merger MS. Historic data may be used if MS have been volatile, for instance when the market is characterised by large, lumpy orders\sn{\Note{Changes in historic MS may provide useful information about the competitive process and the likely future importance of the various competitors, for instance, by indicating whether firms have been gaining or losing MS. }}. 
            
            In any event, the Commission interprets MS in the light of likely market conditions, e.g. if the market is highly dynamic in character and if the market structure is unstable due to innovation or growth

        \subsubsection{Herfindahl-Hirschman Index}

            The overall concentration level in a market may provide useful information about the competitive situation post merger and the structural impact. To measure concentration levels, the EU Commission often applies the \textbf{Herfindahl-Hirschman Index} (HHI). 
            
            The HHI is calculated by summing the squares of the individual MS of all the firms in the market\sn{\Note{Although it is best to include all firms in the calculation, lack of information about very small firms may not be important because such firms do not affect the HHI significantly. Thus, the so-called “truncated HHI” can be used.}}.

            The Herfindahl-Hirschman Index (HHI) gives proportionately greater weight to the market shares (MS) of the larger firms.

            \begin{itemize}
                \item A market with 4 equal-sized firms has an HHI of \( 2,500 \) (\( 25^2 + 25^2 + 25^2 + 25^2 \)).
                \item If \( A = 50\% \), \( B = 25\% \), \( C = 15\% \), and \( D = 10\% \), the HHI is \( 3,450 \).
            \end{itemize}
            
            \Remark{While the absolute level of the HHI provides an indication of the competitive pressure in the market post-merger, the change in the HHI (known as the \textit{delta}) is a useful proxy for the change in concentration directly brought about by the merger. }

            The Herfindahl-Hirschman Index (HHI) ranges from a maximum of 10,000 points (pure monopoly) to a number approaching 0 (atomistic market). Based on economic theory and enforcement experience, the EU Horizontal Merger Guidelines classify markets into three categories:

            \begin{itemize}
                \item \textbf{Unconcentrated Markets:} \( \text{HHI} < 1,000 \) points
                \item \textbf{Moderately Concentrated Markets:} \( 1,000 < \text{HHI} < 2,000 \) points
                \item \textbf{Highly Concentrated Markets:} \( \text{HHI} > 2,000 \) points
            \end{itemize}
            
            The Commission employs the following general standard. Let \( \Delta \text{HHI} \) represent the increase in HHI due to the merger. If:
            
            \begin{enumerate}[label=\alph*.]
                \item \( \Delta \text{HHI} \) is any value, but the post-merger \( \text{HHI} < 1,000 \);
                \item \( \Delta \text{HHI} < 250 \) and \( 1,000 < \text{HHI} < 2,000 \);
                \item \( \Delta \text{HHI} < 150 \) and \( \text{HHI} > 2,000 \),
            \end{enumerate}
            
            then the Commission is unlikely to identify competition concerns, i.e., the opening of an in-depth investigation is unlikely.

        \subsubsection{Exceptions}

            \begin{enumerate}[label=(\alph*)]
                \item A merger involves a potential entrant or a recent entrant with a small market share;
                \item One or more merging parties are important innovators in ways not (or not yet) reflected in market shares;
                \item There are significant cross-shareholdings among the market participants;
                \item One of the merging firms is a maverick firm with a high likelihood of disrupting coordinated conduct;
                \item Indications of past or ongoing coordination, or facilitating practices, are present;
                \item One of the merging parties has a pre-merger market share of 50\% or more.
            \end{enumerate}

\newpage
    \subsection{Unilateral Effects}\label{ToH1}

        A merger may create the opportunity for a unilateral anticompetitive effect. This type of harm is most obvious in the case of a merger to monopoly.

        A merger may also allow a unilateral price increase in markets where the merging firms sell products that customers believe are particularly close substitutes. After the merger, the firm may be able to raise prices profitably without losing many sales\sn{\Note{Such a price increase will be profitable for the merged firm if a sufficient portion of customers would switch between its products rather than switch to products of other firms, and other firms cannot reposition their products to entice customers away.}}.
        
        The most direct effect of the merger will be the loss of competition between the merging firms. If prior to the merger one of the merging firms had raised its price, it would have lost some sales to the other merging firm. The merger removes this constraint.
        
        \Remark{Non-merging firms in the same market can also benefit from the reduction of competitive pressure that results from the merger, since the merging firms' price increase may switch some demand to the rival firms, which, in turn, may find it profitable to increase their prices.}
        
        The reduction in these competitive constraints (see next slide) could lead to significant price increases in the relevant market.

        \subsubsection{Diversion Ratio}

            \Definition{The Diversion Ratio (DR) indicates the fraction of sales lost due to a price increase that would be \textit{recovered} by the merged entity.}{Diversion Ratio}
    
            DR can be measured through:
            \begin{itemize}
                \item Surveys,
                \item Econometric analysis,
                \item Market shares (e.g., logit models).
            \end{itemize}
            
            \textbf{Rule of Thumb:} The merger is considered problematic if \( DR > 20\% \). However, if margins are high, even a lower DR can be problematic.

        \subsubsection{Mergers in Local Markets}

            \begin{itemize}
                \item Some mergers concern companies that compete \textit{locally}, e.g., in retail markets such as grocery shops, banks, perfumeries, pharmacies, and petrol stations.
                \item \textbf{Local competition is important:}
                \begin{itemize}
                    \item Prices are often set locally.
                    \item In some cases, prices are set centrally and uniformly across shops, but certain parameters of retail supply are determined locally.
                    \item Overlaps in many local markets could degrade some competition parameters set centrally and uniformly.
                \end{itemize}
                \item \textbf{Framework of Analysis:} Based on catchment areas (firm-centric approach).
                \item The competitive effects of the merger are assessed based on the merged firm's incentives to increase prices.
            \end{itemize}

    \setcounter{subsection}{1}
    \subsection{(\textit{bis}) Coordinated Effects}\label{ToH2}

        Coordinated effects are likely when the remaining firms can tacitly or explicitly coordinate on price, output, capacity, or other competitive dimensions. Specifically:
        
        \begin{enumerate}
            \item \textbf{To Allow Coordination:} 
            \begin{itemize}
                \item In certain markets, the structure may make it:
                \begin{enumerate}
                    \item \textit{Possible},
                    \item \textit{Economically rational}, and
                    \item \textit{Preferable} 
                \end{enumerate}
                for firms to engage in behaviour aimed at selling at increased prices.\sn{\Note{A merger may significantly impede effective competition through the creation or strengthening of a collective dominant position. This increases the likelihood of firms coordinating behaviour to raise prices without explicit agreements or concerted practices (as per Article 101).}}

            \end{itemize}
            
            \item \textbf{To Further Facilitate Coordination:}
            \begin{itemize}
                \item A merger may make pre-existing coordination\sn{\Note{This can occur by making coordination more robust or enabling firms to coordinate on even higher prices.}}:
                \begin{enumerate}
                    \item Easier,
                    \item More stable, or
                    \item More effective.
                \end{enumerate}
            \end{itemize}
            
            \item \textbf{Conditions for Successful Coordination:}
            Successful coordination typically requires:
            \begin{enumerate}
                \item Reaching or identifying a tacit or explicit agreement that is profitable for all participants.
                \item Means to detect deviations (cheating) from the plan or optimal equilibrium.
                \item Ability to punish cheaters and reinstate the agreement.
                \item Reactions from outsiders, such as non-participating competitors and customers, must not jeopardize expected results.
            \end{enumerate}
        \end{enumerate}
        
        \textbf{Economic Environment Favorable to Coordination:}
        Industrial Organization suggests these conditions are more easily satisfied when:
        \begin{itemize}
            \item The economic environment is simple, stable, and transparent.
            \item The market has few firms.
            \item Products are homogenous.
            \item Price and demand conditions are stable.
            \item The market is characterized by little or no innovation.
            \item It is easy to share markets by allocating customers.
            \item Firms are symmetric in terms of:
            \begin{itemize}
                \item Cost structures,
                \item Market shares,
                \item Capacity, and
                \item Level of vertical integration.
            \end{itemize}
            \item \textit{Analogy:} When two fighters have equal strength and power, they often avoid fighting because they know the risks involved.
        \end{itemize}
\newpage
    \subsection{Countervailing Power}\label{CBP}

        The competitive pressure on a supplier can arise from its customers. Even firms with very high market shares (\textit{MktSh}) may not significantly impede effective competition if their customers possess countervailing buyer power. This refers to the bargaining strength that a buyer has relative to the seller in commercial negotiations, influenced by factors such as:

        \begin{itemize}
            \item The buyer's size,
            \item Its commercial significance to the seller, and
            \item Its ability to switch to alternative suppliers.
        \end{itemize}
        
        A customer may possess countervailing buyer power if it can credibly threaten to resort, within a reasonable timeframe, to alternative sources of supply in response to a supplier's price increase or a deterioration in quality or delivery conditions. This would be the case if the buyer could:
        
        \begin{enumerate}
            \item Immediately switch to other suppliers,
            \item Credibly threaten to vertically integrate into the upstream market, or
            \item Sponsor upstream expansion or entry by:
            \begin{itemize}
                \item Persuading a potential entrant to enter the market, and
                \item Committing to placing large orders with the new entrant.
            \end{itemize}
        \end{enumerate}

    \subsection{Entry and Barriers}\label{PC}

        If a concentration creates opportunities to raise prices above the competitive level, other firms may be enticed to enter the market post-merger.

        \begin{itemize}
            \item Entry—if timely, likely, and sufficient—may counteract the harmful effects of the merger, potentially making enforcement action unnecessary. Under certain conditions, even the possibility of new firms entering the market can keep prices in check.
            \item Many factors can impede entry, such as licensing restrictions, zoning regulations, patent rights, inadequate supply sources, and high costs of capital.
            \item Entry may take a long time, during which consumers would continue paying higher prices. New firms may also fail to attract customers from established firms, particularly in markets where existing players have a proven track record.
        \end{itemize}

        \Remark{Assessing entry conditions is a critical part of the analysis. The likely anticompetitive effects of a merger may be significantly reduced if entry is possible and easy. However, evaluating entry conditions requires intensive fact-finding and is often unique to each industry.}

    \subsection{Efficiencies}\label{ED}

        \begin{itemize}
            \item Concentrations may produce savings by allowing firms to reduce costs, eliminate duplicate functions, or achieve scale economies. Firms may pass merger-specific benefits on to consumers in the form of lower prices, better products, better services, or more choice. The Commission is unlikely to challenge mergers when the efficiencies prevent any potential harm that might otherwise arise from the proposed merger.
            
            \item Theoretical (and above all, only claimed) cost savings would not be enough, however: they must be demonstrated. And the efficiencies must involve a genuine increase in productivity.
        \end{itemize}
        
        In order to be taken into account, efficiencies must fulfill strict conditions, namely:
        
        \begin{itemize}
            \item they must be verifiable (such as that the authority can be reasonably certain that they will materialize and be substantial enough).
            
            \item the efficiencies must be merger specific (i.e., they cannot be achieved by other means than by a merger).
            
            \item the efficiencies must be likely passed on to consumers, and not only recapped by the merging companies alone.
        \end{itemize}
        
        Obviously, efficiencies must at least balance the (excessive) negative effects in terms of increased market power. \textit{Id est}: after having considered the positive effects of the claimed efficiencies, the authorities should conclude that the merger is not able to substantially lessen competition any more.

    \subsection{Failing Firm Defence}\label{FFD}

        We are in a \textit{failing firm} scenario, i.e., the acquired firm would in the near future be forced out of the market because of financial difficulties if not taken over by another firm.
        
        Thus, firstly, the counterfactual analysis changes. The comparison will be between:
        \begin{enumerate}
            \item A future scenario with the two merged entities, and
            \item The countervailing scenario absent the merger, and the failing firm that actually exited the market.
        \end{enumerate}
        
        This kind of merger can be cleared (notwithstanding the impediment to competition) \textbf{IFF}:
        \begin{enumerate}
            \item As said, the firm is failing and in the near future will exit the market;
            \item There is no less anti-competitive alternative than the notified merger (e.g., another merger offer by an out-of-area competitor or a smaller company; an alternative way of saving the company—debt restructuring, etc.);
            \item In the absence of the merger, not only the failing firm but also its assets would inevitably exit the market.
        \end{enumerate}
        
        \Remark{
        Exception\(\Rightarrow\) \textbf{efficiency} (if the acquiring firm is the only one that can use efficiently the assets of the failing firm, thus we can apply \ref{ED}.
        }

\section{Non Horizontal Mergers}

    Non-horizontal mergers can be of two types:
    \begin{itemize}
        \item[(a)]

            \Definition{
            These involve companies operating at different levels of the supply chain. For example, when a manufacturer of a certain product (the ‘upstream firm’) merges with one of its distributors (the ‘downstream firm’), this is called a vertical merger.
            }{Vertical merger}
            
        \item[(b)]

            \Definition{
            These are mergers between firms that are in a relationship which is neither horizontal (as competitors in the same relevant market) nor vertical (as suppliers or customers). In practice, the focus of the analysis is on mergers between companies that are active in closely related markets (e.g., mergers involving suppliers of complementary products or products that belong to the same product range).
            }{Conglomerate merger}
        
    \end{itemize}

    \noindent
    Non-horizontal mergers are less likely to significantly impede effective competition than horizontal ones:
    \begin{itemize}
        \item First, no loss of direct competition occurs. As a result, the main source of anti-competitive effects in horizontal mergers is absent from vertical and conglomerate mergers.
        \item Second, there is more space for efficiencies. For example:
        \begin{itemize}
            \item Integration may decrease transaction costs and allow for better coordination in terms of product design, the organization of the production process, and the way in which the products are sold.
            \item Mergers involving products in a portfolio or range generally sold to the same set of customers (whether complementary or not) may give rise to customer benefits such as one-stop shopping.
        \end{itemize}
    \end{itemize}

        \subsubsection{Competitive Risks}
            The competitive risks of non-horizontal mergers include:
            \begin{itemize}
                \item \textbf{Foreclosure:} This could involve higher access or distribution costs for competitors. Merging firms must gain both the \textit{ability} and \textit{incentive} to foreclose competitors:
                \begin{itemize}
                    \item \textbf{Ability:} Goods are critical for the production of downstream products due to their importance, significant cost factor, or market power.
                    \item \textbf{Incentive:} For example, if the upstream market is a monopoly and the monopolist acquires a downstream competitor, this could create a downstream monopoly. However, if the upstream market was already selling at monopoly prices, profits might not increase, leading to uncertainty about overall effects.
                \end{itemize}
                \item \textbf{Facilitation of coordinated effects:} For instance, controlling a distribution outlet may allow direct control over resale prices, making it easier to coordinate behavior with vertically integrated rivals.
            \end{itemize}


        \subsubsection{Conglomerate Mergers: Tying and Bundling}
            \Example{
            An example involves Coca-Cola and Schweppes:
            \begin{itemize}
                \item Coca-Cola sought to acquire distribution rights for Schweppes in Europe, allowing it to bundle a complete range of products.
                \item This strategy could lead to exclusive distribution agreements, reducing Pepsi's access to downstream markets.
            \end{itemize}
            }
            
            The Commission is unlikely to find concern in non-horizontal mergers where:
            \begin{itemize}
                \item The post-merger market share in each market is below 30\%.
                \item The post-merger Herfindahl-Hirschman Index (HHI) is below 2,000.
            \end{itemize}

        \subsubsection{Exceptions}
            
            Exceptions include special circumstances such as:
            \begin{itemize}
                \item A company likely to expand significantly in the near future due to recent innovation.
                \item Significant cross-shareholdings or cross-directorships.
                \item A firm likely to disrupt coordinated conduct.
                \item Indications of past or ongoing coordination.
            \end{itemize}
            These thresholds are used as initial indicators and do not constitute legal presumptions.

    \subsection{Incentives}

        \Example{
        In October 2022, the Commission prohibited the acquisition of GRAIL by Illumina under the EU Merger Regulation:
        \begin{itemize}
            \item Illumina, a US-based genomics company, develops and sells NGS systems for applications like oncology.
            \item GRAIL, also US-based, develops blood-based cancer tests using Illumina's NGS technology. Its flagship product is \textit{Galleri}, a multi-cancer detection test.
        \end{itemize}
        The Commission's findings included:
        \begin{itemize}
            \item Illumina would have the \textbf{ability to foreclose} GRAIL's rivals, as they rely on Illumina's NGS systems. Alternatives are currently not viable due to high barriers like intellectual property risks and costly switching processes.
            \item Illumina would have \textbf{incentives to foreclose} GRAIL's rivals, given the potential market size of EUR 40 billion by 2035 and the ongoing competition in early cancer detection tests.
        \end{itemize}
        }



